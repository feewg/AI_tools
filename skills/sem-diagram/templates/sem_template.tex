% SEM图LaTeX模板
% 使用方法: 将此模板复制到你的LaTeX文档中,根据需要修改

\documentclass{article}
\usepackage{tikz}
\usetikzlibrary{shapes, arrows.meta, positioning}
\usepackage{amsmath}

% 页面设置
\usepackage[margin=1in]{geometry}

% 中文支持(如需要)
% \usepackage{ctex}

\begin{document}

% ============================================================================
% TikZ样式定义
% ============================================================================

\tikzstyle{latent} = [ellipse, draw, minimum width=2cm, minimum height=1cm]
\tikzstyle{manifest} = [rectangle, draw, minimum width=1.5cm, minimum height=0.8cm]
\tikzstyle{error} = [circle, draw, minimum size=0.6cm]
\tikzstyle{const} = [triangle, draw, minimum size=0.6cm]
\tikzstyle{reg} = [->, >=stealth]
\tikzstyle{cov} = [<->, >=stealth]
\tikzstyle{fixed} = [->, >=stealth, dashed]

% ============================================================================
% 在此处插入你的SEM图代码
% ============================================================================

% 示例:单因子CFA
\section*{单因子CFA模型}

\begin{figure}[h]
\centering
\begin{tikzpicture}
  % 潜变量
  \node[latent] (F1) at (0,2) {$F_1$};
  
  % 观测变量
  \node[manifest] (y1) at (-2,0) {$y_1$};
  \node[manifest] (y2) at (0,0) {$y_2$};
  \node[manifest] (y3) at (2,0) {$y_3$};
  
  % 误差项
  \node[error] (e1) at (-2,-1) {$\varepsilon_1$};
  \node[error] (e2) at (0,-1) {$\varepsilon_2$};
  \node[error] (e3) at (2,-1) {$\varepsilon_3$};
  
  % 路径
  \draw[reg] (F1) -- (y1);
  \draw[reg] (F1) -- (y2);
  \draw[reg] (F1) -- (y3);
  
  \draw[reg] (e1) -- (y1);
  \draw[reg] (e2) -- (y2);
  \draw[reg] (e3) -- (y3);
\end{tikzpicture}
\caption{单因子CFA模型}
\end{figure}

% ============================================================================
% 更多示例模板
% ============================================================================

% 示例:中介模型
\section*{中介模型}

\begin{figure}[h]
\centering
\begin{tikzpicture}
  \node[latent] (X) at (0,1) {$X$};
  \node[latent] (M) at (3,1) {$M$};
  \node[latent] (Y) at (6,1) {$Y$};
  
  \draw[reg] (X) -- node[above] {$a$} (M);
  \draw[reg] (M) -- node[above] {$b$} (Y);
  \draw[reg, bend left] (X) to node[above] {$c'$} (Y);
\end{tikzpicture}
\caption{中介模型}
\end{figure}

% 示例:交叉滞后模型(CLPM)
\section*{交叉滞后模型}

\begin{figure}[h]
\centering
\begin{tikzpicture}
  % T1
  \node[latent] (X1) at (0,2) {$X_1$};
  \node[latent] (Y1) at (3,2) {$Y_1$};
  
  % T2
  \node[latent] (X2) at (0,0) {$X_2$};
  \node[latent] (Y2) at (3,0) {$Y_2$};
  
  % 自回归
  \draw[reg] (X1) -- (X2);
  \draw[reg] (Y1) -- (Y2);
  
  % 交叉滞后
  \draw[reg] (X1) -- (Y2);
  \draw[reg] (Y1) -- (X2);
  
  % 同期相关
  \draw[cov] (X1) -- (Y1);
  \draw[cov] (X2) -- (Y2);
\end{tikzpicture}
\caption{交叉滞后模型(CLPM)}
\end{figure}

% 示例:潜增长模型(LGCM)
\section*{潜增长模型}

\begin{figure}[h]
\centering
\begin{tikzpicture}
  % 潜因子
  \node[latent] (i) at (0,2) {$i$};
  \node[latent] (s) at (3,2) {$s$};
  
  % 观测变量
  \node[manifest] (y1) at (0,0) {$y_1$};
  \node[manifest] (y2) at (1,0) {$y_2$};
  \node[manifest] (y3) at (2,0) {$y_3$};
  \node[manifest] (y4) at (3,0) {$y_4$};
  
  % 载荷
  \draw[reg] (i) -- node[left] {$1$} (y1);
  \draw[reg] (i) -- (y2);
  \draw[reg] (i) -- (y3);
  \draw[reg] (i) -- (y4);
  
  \draw[reg] (s) -- node[left] {$0$} (y1);
  \draw[reg] (s) -- node[left] {$1$} (y2);
  \draw[reg] (s) -- node[left] {$2$} (y3);
  \draw[reg] (s) -- node[left] {$3$} (y4);
  
  % 协方差
  \draw[cov] (i) -- (s);
\end{tikzpicture}
\caption{潜增长模型(LGCM)}
\end{figure}

\end{document}
